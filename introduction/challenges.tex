\section{Challenges}

In addition to the challenges posed by the dynamic features of Javascript, there are several other features that increase the complexity of conducting static analysis on Javascript code.

\subsection{First Class Functions}

Functions in Javascript are treated as \textit{first class citizens}\parencite{firstClassFunctions}, similar to other primitive types such as \textit{number} and \textit{string}. Function expressions support the following behaviour:

\begin{itemize}
    \item{Assignment to variables.}
    \item{Passed as arguments into functions. This is commonly referred to as a \textit{callback}.}
    \item{Returning a function definition from an outer function. This is commonly used in \textit{Higher Order Functions}\parencite{functionalJs4} where an outer function (considered the higher order) returns an inner function to be invoked later on.} 
\end{itemize}

In order to account for this behaviour of functions, we need to conduct data flow analysis to locate the position of functions.

\subsection{Dynamic Object Access}

Objects properties in Javascript can be accessed using the dot \textit{"."} notation or braces \textit{"[ ]"} notation. The difference between the two is that the dot notation requires that a valid identifier\parencite{identifier} be used directly while the braces notation will first try to evaluate the expression within the braces to yield a computed identifier.

\lstinputlisting{./code_examples/ch01/objectAccess.js}

In the example above, \textit{obj} has a member function which returns 1. All three expressions on Lines 11, 13, 15 are valid ways of accessing the member function \textit{a}.

There is no feasible way to determine the return value of \textit{getBChar} at compile time without resorting to heuristics. As such, we will ignore dynamic object access using braces for this project.
