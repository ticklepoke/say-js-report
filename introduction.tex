\chapter{Introduction}

\section{Background}

\section{Dynamic Languages}

Dynamic languages can be generalised as languages that differ several beheviours of the language to runtime, that would otherwise occur during compile time for static languages.\footnote{\url{https://developer.mozilla.org/en-US/docs/Glossary/Dynamic_programming_language}}

\section{Extended Call Graphs}

Call graphs provide a visualisation of the relations between functions in a codebase. When a function \textit{foo} calls another function \textit{bar} within its body, we can describe this process as \textit{foo} invoking \textit{bar}.

\lstinputlisting{./code_examples/fooBar.js}

In a complex codebase with a large number of invocations, a graph representing these invocations would be useful in helping a developer to quickly understand the codebase.
\\
Call graphs with function - function relations are defined as follows:

\begin{definition}
    A call-graph $G(V, E)$ consists of the vertex set $V$ representing functions and edge set $E$ representing invocations such that $e_{ij}\in{E}$ is a directed edge from $v_i\in{V}$ to $v_j\in{V}$ if function $i$ invokes function $j$. 
\end{definition}

It would be helpful for the developer if a similar relation can be used to visualise when functions refer to variables outside its scope.

\lstinputlisting{./code_examples/fooVar.js}

This project attempts to extend the call graph by adding function - variable relations, resulting in the following definition:

\begin{definition}
    An extended call-graph $G'(V',E')$ consists of the vertex set $V'$ representing functions and variables, and the edge set $E'$ representing invocations and references such that $e_{ij}\in{E'}$ is a directed edge from $v_i\in{V'}$ to $v_j\in{V'}$ if function $i$ invokes function $j$ or if function $i$ references variable $j$.
\end{definition}

\section{Applications}

\section{Motivating Example}

\section{Challenges}


